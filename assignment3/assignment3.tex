\documentclass[10pt]{article}
\usepackage{fullpage,enumitem,amsmath,amssymb,graphicx}
\usepackage{tikz}
\usepackage{verbatim}

\begin{document}

\begin{center}
{\Large CS224N Winter 2016 Homework [3]}

\begin{tabular}{rl}
SUNet ID: & [jiajuns, xxx, xxx] \\
Name: & [Jiajun Sun, Sijun He, Mingxiang Chen] \\
\end{tabular}
\end{center}

By turning in this assignment, I agree by the Stanford honor code and declare
that all of this is my own work.

\section*{Problem 1}
\begin{enumerate}[label=(\alph*)]
\item
i.\\
Yesterday, price of Apple increased 6.56 percent.\\
Here apply can be interpreted as Apple company or the apple we eat.\\
\\
Mogen Stanley has bought a land besides Sand Hill Road as their bay area headquater.
Mogen Stanley can be interpreted as a name or a company.\\
\\

ii.\\
Words alone can have ambiguity, therefore adding features other than the word itself can help reduce ambiguity.\\
\\

iii.\\
First, the context of the word can help. The words around can imply the correct meaning of the center word.
Second, dependence structure could give the information on what is the function of the center word in the sentences.
This information can help interprete its entity.

\item
i.\\
$e^{(t)}$ has dimension of $1 \times 2(w+1)D$\\
$W$ has dimension of $2(w+1)D \times H$\\
$U$ has dimension of $H \times C$\\
\\
ii.\\
$$cost(ReLu) = o(H) + o(2(w+1)D \times H)$$
$$cost(softmax) = o(H \times C) + o(C)$$
$$cost(CE) = o(C)$$
Given the sentences has length of $T$, the computation complexity is:
$$
\begin{aligned}
cost
& = T \times o(o(H) + o(2(w+1)D \times H) + o(H \times C) + o(C) + o(C))\\
& \approx T o(2(w+1)D \times H)\\
& \approx T o(wDH)
\end{aligned}
$$

\item
(code)

\end{enumerate}


\end{document}

